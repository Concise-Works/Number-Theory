

\section{The Chinese Remainder Theorem}

\begin{Note}
    \textbf{Note:} $\mathbb{Z}^+$ denotes the set of positive integers, and $\{x_i\}_{i=1}^k$ is short for $\{x_1,...,x_k\}$.
\end{Note}
\begin{theo}[Chinese Remainder Theorem (CRT)]

    Let $\{n_i\}_{i=1}^k\in\mathbb{Z}^+$ all be coprime to each other and let $\{a_i\}_{i=1}^k$ be arbitrary integers. Then there is a solution $a\in\mathbb{Z}$ to the system of congruences:
    \begin{align*}
        a &\equiv a_1 \pmod{n_1} \\
        a &\equiv a_2 \pmod{n_2} \\
        &\hspace*{.5em}\vdots \\
        a &\equiv a_k \pmod{n_k}
    \end{align*}
    
    
    \noindent
    Moreover, if $a$ and $b$ are solutions to the system, then $a\equiv b\pmod{\prod_{i=1}^k n_i}$.
\end{theo}
\begin{Proof}[Solving a System of Congruences (Part 1)]
Let $\{n_i\}_{i=1}^k\in\mathbb{Z}^+$ all be pairwise coprime, and let $\{a_i\}_{i=1}^k$ be arbitrary integers,\\

\noindent
\textbf{Existence:} (i) Construct a partial solution for each congruence. (ii) Each
partial solution must not interfere with other congruences. (iii) Combine partial solutions:\\

\noindent
We define indexes $i,j=1,...,k$ representing any two $e_1,...,e_k$ integers such that:
 \[e_j \equiv
 \begin{cases} 
 1 \pmod{n_i} & \text{if } j = i, \text{ (target congruence)}\\
 0 \pmod{n_i} & \text{if } j \neq i \text{ (non-interfering)}.
 \end{cases}
 \]
 I.e., $e_j$ has multiplicative identity to it's own system, and additive identity to all other systems
 by being some multiple. This allows us to construct:
 \begin{align*}
    e_1\cdot a_1 &\equiv 1\cdot a_1 \pmod{n_1} \\
    e_2\cdot a_2 &\equiv 1\cdot a_2 \pmod{n_2} \\
    &\hspace*{.5em}\vdots \\
    e_k\cdot a_k &\equiv 1\cdot a_k \pmod{n_k}
\end{align*}
Using additive identity, we close partial-solutions to $a = \sum_{i=1}^k e_ia_i$, the whole solution.
\end{Proof}

\newpage

\begin{Proof}[Solving a System of Congruences (Part 2)]

\noindent
To construct such $e_1,...,e_k$, let $n := \prod_{i=1}^k n_i$ (the product of all moduli) and $n_i^* := n/n_i$. Then, $n_i$ and $n_i^*$ are coprime, meaning they have solution $n_i^*z\equiv 1\pmod{n_i}$ for some $z\in\mathbb{Z}$.\\

\noindent
Then $z=(n_i^*)^{-1}$, we can now define $e_i := n_i^*z$ for each $i=1,...,k$. Therefore, $e_i\equiv 1\pmod{n_i}$.\\
Since $n$ contains shared factors, and we take $n_i$ at congruence $i$, $e_i\equiv 0\pmod{n_j}$ for $i\neq j$.\\

\noindent
Thus, we can now construct the solution $a = \sum_{i=1}^k e_ia_i$.

\end{Proof}
\begin{Proof}[Uniqueness of Solutions (Part 3)]

    If \( a \) and \( a' \) both satisfy the system of congruences
    \[
    a \equiv a_i \pmod{n_i} \quad \text{and} \quad a' \equiv a_i \pmod{n_i} \quad \text{for } i = 1, \dots, k
    \]
    Then they must be congruent, i.e., \( a \equiv a' \pmod{\prod_{i=1}^k n_i} \).\\
    
    \noindent
\end{Proof}
\textbf{Example:} We'll find solution $a$ to the system of congruences:
\begin{align*}
    a &\equiv 3 \pmod{5} \\
    a &\equiv 5 \pmod{7} \\
    a &\equiv 2 \pmod{11}
\end{align*}
Observe that ($3$ mod $5)=\{3,8,13,18,23,28,33,...\}$, and ($5$ mod $7)=\{5,12,19,26,33,...\}$.
Sets describing $3+5k$ and $5+7k$ respectively. We see,  $3\equiv 33\pmod{5}$, and $5\equiv 33\pmod{7}$.\\

\noindent
Obtaining $e_1=7$, as $3+5(7)\Longrightarrow3(7)\equiv 1\pmod{5}$, and $5+7(7)\Longrightarrow5(7)\equiv0\pmod{7}$. 

\noindent
We can take $n=5\cdot7=35$ to construct a new system:
\begin{align*}
    a &\equiv 33 \pmod{35} = 33+35k = \{33,68,...\} \\
    a &\equiv 2 \pmod{11} = 2+11k = \{2,13,24,35,46,57,68,...\}
\end{align*}
We see that $33\equiv 68\pmod{35}$, and $2\equiv 68\pmod{11}$. Thus, $a=68$:
\begin{align*}
    68 &\equiv 3 \pmod{5} \\
    68 &\equiv 5 \pmod{7} \\
    68 &\equiv 2 \pmod{11}
\end{align*}
We can design a general algorithm based off this example to solve such systems.

\newpage
\subsection{CRT Algorithm}
\begin{Func}[CRT Algorithm - \texttt{crt()}]
Computes \( a\in\mathbb{Z} \) satisfying a given system of congruences:

\begin{algorithm}[H]
    \SetAlgoLined
    \KwIn{Positive integers \( \{n_i\}_{i=1}^k \) and integers \( \{a_i\}_{i=1}^k \)}
    \KwOut{An integer \( a \) satisfying the system of congruences}
    \vspace{.5em}
    \SetKwProg{Fn}{Function}{:}{\KwRet{\( a \)}}
    \Fn{\texttt{crt}(\( \{n_i\}_{i=1}^k, \{a_i\}_{i=1}^k \))}{
        \( a \gets a_1 \)\;
        \( N \gets n_1 \)\;
        \For{\( i \gets 2 \) \KwTo \( k \)}{
            \While{\( a \mod n_i \neq a_i \)}{
                \( a \gets a + N \)\;
            }
            \( N \gets N \times n_i \)\;
        }
    }
\end{algorithm}
\end{Func}
We compute just like the example above:
\begin{enumerate}
    \item First take $a_1$ and $n_1$ as our initial solution and modulus.
    \item Then iterate starting with $a_2$ and $n_2$ to find solution $a$ mod $(n_i)$ = $a_i$.
    \item If $a$ is not congruent to $a_i$, we increment $a$ by $N$ until it is.
    \item Then update $N$ to the new product, and move to the next congruence.
\end{enumerate}
% We avoid computing inverse $e_i=\left(\dfrac{n}{n_i}\right)^{-1}$ for each $a_i$ as  

    