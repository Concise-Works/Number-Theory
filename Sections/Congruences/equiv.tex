\section{Equivalence Relations}

\begin{Def}[Equivalence Relation]

    An \textbf{equivalence relation} on set $S$ is a relation $\sim$ which satisfies:
    \begin{enumerate}
        \item \textbf{Reflexivity:} For all $a \in S$, $a \sim a$.
        \item \textbf{Symmetry:} For all $a, b \in S$, if $a \sim b$, then $b \sim a$.
        \item \textbf{Transitivity:} For all $a, b, c \in S$, if $a \sim b$ and $b \sim c$, then $a \sim c$.
    \end{enumerate}
    With $a\sim a$ reading, ``$a$ is related to $a$.''
\end{Def}


\begin{Def}[Equivalence Class]

    \label{def:equiv_class}

    For $\sim$ equivalence relation on set $S$. For each $a \in S$, the \textbf{equivalence class} of $a$ is the set
    \[
        [a] = \{x \in S \mid x \sim a\}.
    \]
\end{Def}

\begin{Note}
    \textbf{Note:} For $x\in[a]$, $x$ is a \textbf{representative} of the equivalence class $[a]$ (\ref{def:representative}).
\end{Note}

\begin{theo}[Equivalence Class Uniqueness]

    For $\sim$ equivalence relation on set $S$, for all $a, b \in S$:

    \begin{itemize}
        \item[(i)]  $a \in [a]$.
        \item[(ii)] $a \in [b] \Longrightarrow [a] = [b]$.
    \end{itemize}
    
\end{theo}

\newpage

\begin{Proof}[Equivalence Class Uniqueness]
    For $a,b \in S$:
        \begin{itemize}
            \item[(i)]  Since $\sim$ is reflexive, $a \sim a$.
            \item[(ii)] Suppose $a \in [b]$. Then $a \sim b$. Then for $x \in S$,
                        \begin{align*}
                            x \in [a]&\Longrightarrow x \sim a \text{ (Definition of $[a]$ (\ref{def:equiv_class}))} \\
                            &\Longrightarrow x \sim b \text{ (Transitivity, $x\sim a\land a\sim b$)} \\
                            &\Longrightarrow x \in [b] \text{ (Definition of $[b]$ (\ref{def:equiv_class}))}
                        \end{align*}
                        Thus $[a] \subseteq [b]$. Similarly, $[b] \subseteq [a]$. Therefore $[a] = [b]$.
        \end{itemize}
\end{Proof}

\section{Modular Congruences}
Continuing with the notion of residues in, we introduce the concept of modular congruences (\ref{def:residue}). 

\begin{Def}[Modular Congruence]

    \label{def:mod_cong}

    For $n \in \mathbb{N}$, $a, b \in \mathbb{Z}$, $a$ is \textbf{congruent} to $b$ modulo $n$ if $n \mid (a - b)$, denoted as
    \[
        a \equiv b \pmod{n}.
    \]
    If $n\nmid (a-b)$, then $a \not\equiv b \pmod{n}$.
\end{Def}
\noindent
I.e., $a$ and $b$ have the same remainder when divided by $n$.
\begin{Note}
    \textbf{Note:} $a \equiv b \pmod{n}$: $\mathbf{a}$ and $\mathbf{b}$ are \textbf{dividends} of $\mathbf{n}$ our \textbf{divisor}, which relate by \textbf{remainder}.
\end{Note}
\begin{theo}[Modular Congruence Properties]

    For all \(a, b, c \in \mathbb{Z}\), and some positive integer $n$:
\begin{itemize}
    \item[(i)] \( a \equiv a \pmod{n} \);
    \item[(ii)] \( a \equiv b \pmod{n} \Longrightarrow b \equiv a \pmod{n} \);
    \item[(iii)] \( a \equiv b \pmod{n} \) and \( b \equiv c \pmod{n} \Longrightarrow a \equiv c \pmod{n} \).
\end{itemize}

\end{theo}

\newpage

\begin{Proof}[Modular Congruence Properties]
    For all \(a, b, c \in \mathbb{Z}\), and some positive integer $n$:
    \begin{itemize}
        \item[(i)] \( a \equiv a \pmod{n} \) so \( n \mid (a - a) \), which holds.
        \item[(ii)] \( a \equiv b \pmod{n} \) so $n$ divides \((a-b)\) and \(-(a - b)=(b-a) \), then \( b \equiv a \pmod{n} \).
        \item[(iii)] \( a \equiv b \pmod{n} \) so \( n \mid (a - b) \), and \( b \equiv c \pmod{n} \) is  \( n \mid (b - c) \). Therefore,
       \begin{align*}
              &\quad n \mid (a - b) \quad \text{and} \quad n \mid (b - c)& \\
              \Longrightarrow &\quad n \mid [(a - b) + (b - c)]  \\
              \Longrightarrow &\quad n \mid (a - c)  \\
              \Longrightarrow &\quad a \equiv c \pmod{n}.
       \end{align*}
    \end{itemize}
\end{Proof}

\begin{theo}[Modular Arithmetic]

    Let \(a, a', b, b', n \in \mathbb{Z}\) with \(n > 0\). If
    \[
    a \equiv a' \pmod{n} \quad \text{and} \quad b \equiv b' \pmod{n},
    \]
    then
    \[
    a + b \equiv a' + b' \pmod{n} \quad \text{and} \quad a \cdot b \equiv a' \cdot b' \pmod{n}.
    \]
    
\end{theo}

\begin{Proof}[Modular Arithmatic]
    
    \textbf{Addition:} For \(a, a', b, b', n \in \mathbb{Z}\),
    \begin{itemize}
        \item So $a\equiv a'\pmod{n}$ then $n\mid(a-a')$ means $a-a'=nx$ for some $x\in\mathbb{Z}$.
        \item Similarly, $b\equiv b'\pmod{n}$ then $b-b'=ny$ for some $y\in\mathbb{Z}$.
        \item Adding both equations, $(a-a')+(b-b')=(nx+ny)$ so $(a+b)-(a+b')=n(x+y)$.
        \item Therefore, $a+b\equiv a'+b'\pmod{n}$, as $n\mid(a+b)-(a'+b')$.
    \end{itemize}
    \textbf{Multiplication:} Continuing,
    \begin{itemize}
        \item If we multiply both equations, $(a-a')(b-b')=(nx)(ny)$ so $(ab)-(a'b')=n(xy)$.
        \item Therefore, $ab\equiv a'b'\pmod{n}$, as $n\mid(ab)-(a'b')$.
    \end{itemize}
     

\end{Proof}

\newpage

\begin{theo}[Least Residue]

    Let \(a, n \in \mathbb{Z}\) with \(n > 0\). There exists unique $z \in \mathbb{Z}$ such that:
    \begin{itemize}
        \item [(i)] \(0 \leq z < n\),
        \item [(ii)] \(a \equiv z \pmod{n}\).
        \item [(iii)] \(z\) is the \textbf{least residue} of \(a\) modulo \(n\).
    \end{itemize}
    Particularly, for all $x\in\mathbb{Z}$, $z\in[x,x+n)$.
\end{theo}
I.e., the least non-negative remainder $r$, which could be thought of as $r:=a\text{ mod } n$.
\begin{Note}
    \textbf{Note:} The period $[x,x+n)$, contains possible remainders, a call back to the Division Alg. (\ref{theo:div_alg_ext}).
\end{Note}

\begin{Proof}[Least Residue]

    For some $a,q,n,r\in\mathbb{Z}$,\\
    The Division Algorithm grantees existence, for $a=qn+r:0\leq r<n$ (\ref{theo:div_alg_ext}). Residues 
    mod $n>0$ are non-empty. Thus by the Well-Ordering Principle, there's a least element.
\end{Proof}

\noindent
\textbf{Example:} Working to find the \textbf{set of solutions} $\mathbf{z}$ for $a\equiv z\pmod{n}$, i.e., find $z$ that satisfies,
\begin{align*}
3z+4 \equiv 6 \pmod{7} & \text{ (Given)} \\
3z \equiv 2 \pmod{7} & \text{ (Subtracting 4 from both sides)} \\
\end{align*}
We can't necessarily divide, but we can shift residue by some favorable factor.
\begin{align*}
3z\cdot5 \equiv 2\cdot5 \pmod{7} & \text{ (Multiply 5 to both sides)} \\
1\cdot z\equiv 10 \pmod{7} & \text{ (Since $15\equiv1\pmod{7}$)} 
\end{align*}
Finding solution $z\equiv10\pmod{7}$, which we can reduce to $z\equiv 3\pmod{7}$, as $3\equiv10\pmod{7}$.\\
We say ``integers $z$ has solutions'' as $z\in[3]_{7}=\{3+7k:k\in\mathbb{Z}\}$ possible solutions.\\

\begin{Note}
    \textbf{Note:} $[3]_{7}$ reads as \textit{``the residue class 3 modulo 7.''} Mentioned in (\ref{def:equiv_class}).
\end{Note}

\newpage

\section{Solving Linear Congruences}

\begin{theo}[Modular Multiplicative Identities]

    \label{theo:mod_mult_id}
    
Let \(a, n \in \mathbb{Z}\) with \(n > 0\), and let \(d := \gcd(a, n)\).
\begin{itemize}
    \item[(i)] For every \(b \in \mathbb{Z}\), the congruence \(az \equiv b \pmod{n}\) has a solution \(z \in \mathbb{Z}\) if and only if \(d \mid b\).
    
    \item[(ii)] For every \(z \in \mathbb{Z}\), we have \(az \equiv 0 \pmod{n}\) if and only if \(z \equiv 0 \pmod{n/d}\).
    
    \item[(iii)] For all \(z, z' \in \mathbb{Z}\), we have \(az \equiv az' \pmod{n}\) if and only if \(z \equiv z' \pmod{n/d}\).
\end{itemize}
\end{theo}
\begin{Proof}[Linear Congruence Identities]
    Let \(a, n \in \mathbb{Z}\) with \(n > 0\), and let \(d := \gcd(a, n)\).\\

    (i)

    \vspace{-3em}
    \begin{align*}
        &az \equiv b \pmod{n} \quad \text{for some } z \in \mathbb{Z} \\
        \Longleftrightarrow \quad &az - b = ny \quad \text{for some } z, y \in \mathbb{Z} \quad \text{(Def. of congruence (\ref{def:mod_cong}))} \\
        \Longleftrightarrow \quad &az - ny = b \quad \text{for some } z, y \in \mathbb{Z} \\
        \Longleftrightarrow \quad &d \mid b \quad \text{(By Bezout's Identity (\ref{theo:bezouts_identity}))}.
        \end{align*}

    (ii) Above is Bezout's Identity as $a$ and $n$ form a linear combination of $b$.-
    \begin{align*}
        n \mid az \Longleftrightarrow \quad & n/d \mid (a/d)z  \text{ (Props. of Divisibility (\ref{theo:properties_of_divisibility}))}\\
        \Longleftrightarrow \quad & n/d \mid z.  \text{ (Cancellation of GCD: $\gcd(a/d,n/d)=1$ (\ref{theo:cancellation_of_gcd}))}
    \end{align*}


    (iii)

    \vspace{-3em}
    \begin{align*}
        &az \equiv az' \pmod{n} \\
        \Longleftrightarrow \quad & a(z - z') \equiv 0 \pmod{n} \\
        \Longleftrightarrow \quad & z - z' \equiv 0 \pmod{n/d} \quad \text{(By Part (ii))} \\
        \Longleftrightarrow \quad & z \equiv z' \pmod{n/d}.
        \end{align*}

\end{Proof}

\noindent
For emphasis, as we saw above:
\begin{Def}[GCD Reduction]

    For $a,n\in\mathbb{Z}$, $d:=\gcd(a,n)$, then $\gcd(a/d,n/d)=1$.
\end{Def}

\newpage

\begin{Note}
    \textbf{Note:} ``$\rightarrow$'' (Maps to), ``$\mapsto$'' (Defines the action of how a single element maps to another), ``image'' (the set of all outputs), and ``pre-images'' (the set of all inputs).
\end{Note}

\noindent
A corollary to the above theorem (\ref{theo:mod_mult_id}):
\begin{theo}[Modular Multiplicative Map]

    Let $a,n\in \mathbb{Z}$ with $n>0$, and residue classes $I_n:=\{0,\dots,n-1\}$. Then $(a \mod n)\in I_n$.\\
    Notably, for $z\in\mathbb{Z}$, $(az \mod n)$ is also in $I_n$.\\

    \noindent
    I.e., $(az \mod n)$ is some re-ordering of the residue class $(a \mod n)$. Defining function, $\tau_a$:
    \begin{align*}
        \tau_a:&\enspace I_n \rightarrow I_n: z \mapsto az \text{ mod } n. \tag{3.2.1}
    \end{align*}
    The length of the image of $\tau_a$ is the number of distinct factors of $n$ relative to $a$, i.e., $n/d$. Let the image of $\tau_a$ be:
    \begin{align*}
        E := \{az \mod n : z \in I_n\} &= \{i \cdot d \mod n : i = 0, \dots, n/d - 1\}. \tag{3.2.2}
    \end{align*}
    The length of the pre-images of $\tau_a$ is the number of $z$ solutions to $az\equiv b \pmod{n}$, i.e., $d$. Let the pre-images of $\tau_a$ be:
    \begin{align*}
        P := \{z \in I_n : az \equiv b \pmod{n}\}. \tag{3.2.3}
    \end{align*}
    It follows that $\tau_a$ is a bijection (one-to-one and onto) if and only if $\gcd(a, n) = 1$. Then, the length of the image is $n$, and each pre-image has length 1.

\end{theo}
\textbf{Example:} for $a=1,2,3,4,5,6$ and $n=15$,
\begin{table}[h!]
    \setlength{\tabcolsep}{4pt} % Adjust column separation
    \renewcommand{\arraystretch}{1.2} % Adjust row separation
    \centering
    \begin{tabular}{|c|p{.2cm}|p{.6cm}|p{.6cm}|p{.6cm}|p{.6cm}|p{.6cm}|p{.6cm}|p{.6cm}|p{.6cm}|p{.6cm}|p{.6cm}|p{.6cm}|p{.6cm}|p{.6cm}|p{.6cm}|}
        \hline
$z$ & 0 
& \cellcolor{OliveGreen!5}1 
& \cellcolor{OliveGreen!10}2 
& \cellcolor{OliveGreen!15}3 
& \cellcolor{OliveGreen!20}4 
& \cellcolor{OliveGreen!25}5 
& \cellcolor{OliveGreen!30}6 
& \cellcolor{OliveGreen!35}7 
& \cellcolor{OliveGreen!40}8 
& \cellcolor{OliveGreen!45}9 
& \cellcolor{OliveGreen!50}10 
& \cellcolor{OliveGreen!55}11 
& \cellcolor{OliveGreen!60}12 
& \cellcolor{OliveGreen!65}13 
& \cellcolor{OliveGreen!70}14 \\ \hline
        $2z \mod 15$ 
        & 0 
        & \cellcolor{OliveGreen!10}2 
        & \cellcolor{OliveGreen!20}4 
        & \cellcolor{OliveGreen!30}6 
        & \cellcolor{OliveGreen!40}8 
        & \cellcolor{OliveGreen!50}10 
        & \cellcolor{OliveGreen!60}12 
        & \cellcolor{OliveGreen!70}14 
        & \cellcolor{OliveGreen!5}1 
        & \cellcolor{OliveGreen!15}3 
        & \cellcolor{OliveGreen!25}5 
        & \cellcolor{OliveGreen!40}7 
        & \cellcolor{OliveGreen!45}9 
        & \cellcolor{OliveGreen!55}11 
        & \cellcolor{OliveGreen!65}13 \\ \hline
        $3z \mod 15$ 
        & 0 
        & \cellcolor{OliveGreen!15}3 
        & \cellcolor{OliveGreen!30}6 
        & \cellcolor{OliveGreen!45}9 
        & \cellcolor{OliveGreen!60}12 
        & 0 
        & \cellcolor{OliveGreen!15}3 
        & \cellcolor{OliveGreen!30}6 
        & \cellcolor{OliveGreen!45}9 
        & \cellcolor{OliveGreen!60}12 
        & 0 
        & \cellcolor{OliveGreen!15}3 
        & \cellcolor{OliveGreen!30}6 
        & \cellcolor{OliveGreen!45}9 
        & \cellcolor{OliveGreen!60}12 \\ \hline
        $4z \mod 15$ 
        & 0 
        & \cellcolor{OliveGreen!20}4 
        & \cellcolor{OliveGreen!40}8 
        & \cellcolor{OliveGreen!60}12 
        & \cellcolor{OliveGreen!5}1 
        & \cellcolor{OliveGreen!25}5 
        & \cellcolor{OliveGreen!45}9 
        & \cellcolor{OliveGreen!65}13 
        & \cellcolor{OliveGreen!10}2 
        & \cellcolor{OliveGreen!30}6 
        & \cellcolor{OliveGreen!50}10 
        & \cellcolor{OliveGreen!70}14 
        & \cellcolor{OliveGreen!15}3 
        & \cellcolor{OliveGreen!40}7 
        & \cellcolor{OliveGreen!55}11 \\ \hline
        $5z \mod 15$ 
        & 0 
        & \cellcolor{OliveGreen!25}5 
        & \cellcolor{OliveGreen!50}10 
        & 0 
        & \cellcolor{OliveGreen!25}5 
        & \cellcolor{OliveGreen!50}10 
        & 0 
        & \cellcolor{OliveGreen!25}5 
        & \cellcolor{OliveGreen!50}10 
        & 0 
        & \cellcolor{OliveGreen!25}5 
        & \cellcolor{OliveGreen!50}10 
        & 0 
        & \cellcolor{OliveGreen!25}5 
        & \cellcolor{OliveGreen!50}10 \\ \hline
        $6z \mod 15$ 
        & 0 
        & \cellcolor{OliveGreen!30}6 
        & \cellcolor{OliveGreen!60}12 
        & \cellcolor{OliveGreen!15}3 
        & \cellcolor{OliveGreen!45}9 
        & 0 
        & \cellcolor{OliveGreen!30}6 
        & \cellcolor{OliveGreen!60}12 
        & \cellcolor{OliveGreen!15}3 
        & \cellcolor{OliveGreen!45}9 
        & 0 
        & \cellcolor{OliveGreen!30}6 
        & \cellcolor{OliveGreen!60}12 
        & \cellcolor{OliveGreen!15}3 
        & \cellcolor{OliveGreen!45}9 \\ \hline
    \end{tabular}
\end{table}

\begin{itemize}
    \item \textbf{Row:2} We see $2$ and $15$ are coprime, hence $n$ images, $\{0,\dots,n-1\}$.
    \item \textbf{Row:3} We see $3$ and $15$. Taking out common factors, $15/3$, we get $5$ distinct images.
    \item \textbf{Row:4} We see $4$ and $15$ are coprime, hence $n$ images, $\{0,\dots,n-1\}$.
    \item \textbf{Row:5} We see $5$ and $15$. Taking out common factors, $15/5$, we get $3$ distinct images.
    \item \textbf{Row:6} We see $6$ and $15$. Taking out common factors, $15/3$, we get $5$ distinct images.
\end{itemize}

\newpage

Another corollary to the above theorem (\ref{theo:mod_mult_id}):
\begin{theo}[Modular Congruence Cancellation]

    Let \( a, b, c, n \in \mathbb{Z} \) with \( n > 0 \) and \( \gcd(c, n) = 1 \). If \( ac \equiv bc \pmod{n} \), then \( a \equiv b \pmod{n} \).
    
    \end{theo}
    
\noindent
\textbf{Example:} We'll demonstrate different representations of members residue class $[2]_5$:
\begin{align*}
    8 \equiv 13 \pmod{5} & \quad \text{(i)} \\
    2\cdot4 \equiv 3\cdot5 \pmod{5} & \quad \text{(ii)} \\
    2\cdot4 \equiv (-3)\cdot4 \pmod{5} & \quad \text{(iii)} \\ 
    2 \equiv -3 \pmod{5} & \quad \text{(iv)}
\end{align*}
Indeed $2\equiv-3\pmod{5}$, as $2+3\equiv3-3\pmod{5}$. To show this, observe:

\begin{table}[h!]
    \setlength{\tabcolsep}{10pt} % Adjust column separation
    \renewcommand{\arraystretch}{1.2} % Adjust row separation
    \centering
\begin{tabular}{|*{14}{c|}}
    
    
    \hline
   \rowcolor{OliveGreen!10}\cellcolor{white}$a$ & $0$ & $1$ & $2$ & $3$ & $4$ & $5$ & $6$ & $7$ & $8$ & $9$ & $10$ & $11$ & $12$ \\
    \hline
       & $0$ & $1$ & $2$ & $3$ & $4$ & $0$ & $1$ & $2$ & $3$ & $4$ & $0$ & $1$ & $2$ \\
    \cline{2-14}
    \multirow{-2}{*}{$a\mod 5$}&0&-4&-3&-2&-1&0&-4&-3&-2&-1&0&-4&-3\\
    \hline
\end{tabular}
\end{table}
\noindent
\underline{Think of \textbf{negative numbers} as traveling backwards within the residue class.}

\begin{Def}[Modular Iverses]
    
    Let $a,n\in\mathbb{Z}$ with $n>0$. If $az\equiv1\pmod{n}$, then $z$ is the \textbf{modular inverse} of $a$ \textbf{modulo} $n$\\
    and unique.\\

    \noindent
    Denoted: $a^{-1}\pmod{n}$.
\end{Def}
\noindent
If inverse $z$ modulo $n$ exists, it is unique, as if there were another inverse $z'$, then $z'\equiv z\pmod{n}$.\\

\noindent
Restating (\ref{theo:mod_mult_id}) under coprime conditions:
\begin{theo}[Coprime Modular Multiplicative Identities]
    
    \label{theo:co_mod_mult_id}
    
    Let \(a, n \in \mathbb{Z}\) with \(n > 0\), and let \(\gcd(a, n) = 1\).
    \begin{itemize}
        \item[(i)] The congruence \(az \equiv 1 \pmod{n}\) has a solution \(z \in \mathbb{Z}\), the modular inverse.
        
        \item[(ii)] If \(az \equiv 0 \pmod{n}\), then \(z \equiv 0 \pmod{n}\) (i.e., \(z\) must be a multiple of \(n\)).
        
        \item[(iii)] If \(az \equiv az' \pmod{n}\), then \(z \equiv z' \pmod{n}\) (i.e., \(a\) cancels out, as long as \(\gcd(a, n) = 1\)).
    \end{itemize}
  
\end{theo}

\noindent
Try to find inverses from the above table. Take an $a$ and find solution $z$ to $az\equiv1\pmod{5}$.

\newpage









