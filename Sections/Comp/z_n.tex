\newpage 
\section{Computing in \(\mathbb{Z}_n\)}
Recall the Least Residue definition (\ref{theo:least_rep}). 
To stress, if $a\in\Z$, and $\in\mathbb{Z}_n:=\{0,\dots,n-1\}$ Then $a$ is the least residue.
Moreover, $\alpha:=[a]_n$, where $\alpha$ or $a$ are used interchangeably, as they refer to the same element in $\mathbb{Z}_n$.
Let $\beta:=[b]_n\mathbb{Z}_n$, then $\alpha+\beta=[a+b]_n=a+b$ if and only if $a$ and $b$ are the least residues. 

\begin{Func}[Least Residue Representation - \textit{rep()}]

    Given an integer $a \in \mathbb{Z}_n$, \textbf{\textit{rep($a$)}} refers to the least residue representation of $a$ modulo $n$.
\end{Func}

\begin{theo}[Arithmetic Operations in \(\mathbb{Z}_n\)]

    \label{theo:zn_operations}
    Let $\alpha, \beta \in \mathbb{Z}_n$, where $n$ is a modulus. We define the following operations in terms of their least residue representations \textit{rep($\cdot$)}:
    \begin{itemize}
        \item \textbf{Addition:} To compute \(\alpha + \beta\) in \(\mathbb{Z}_n\), we first calculate the integer sum \(\textit{rep}(\alpha) + \textit{rep}(\beta)\), then subtract $n$ if the result is greater than or equal to $n$.
        \item \textbf{Subtraction:} To compute \(\alpha - \beta\) in \(\mathbb{Z}_n\), we first calculate the integer difference\\
         \(\textit{rep}(\alpha) - \textit{rep}(\beta)\), adding $n$ if the result is negative.
        \item \textbf{Multiplication:} To compute \(\alpha \cdot \beta\) in \(\mathbb{Z}_n\), we calculate \(\textit{rep}(\alpha) \cdot \textit{rep}(\beta) \mod n\) using integer multiplication followed by a division with remainder.
    \end{itemize}
    These operations can be performed in time complexities:
    \begin{itemize}
        \item Addition and Subtraction: \(O(||n||)\)
        \item Multiplication, Division, and mod: \(O(||n^2||)\)
    \end{itemize}
\end{theo}

\begin{Def}[Binary Inputs and Memory Allocation]

    Say we want an array of 4 integers, all 4 bytes in size. Each entry is initialized at a specific memory address in hexadecimal. Then, for example:
    \begin{align*}
        [0] &= 0000 \ 0000 \ 0000 \ 0000 \quad \text{at memory address 0x4a0}\\
        [1] &= 0000 \ 0000 \ 0000 \ 0000 \quad \text{at memory address 0x4a4}\\
        [2] &= 0000 \ 0000 \ 0000 \ 0000 \quad \text{at memory address 0x4a8}\\
        [3] &= 0000 \ 0000 \ 0000 \ 0000 \quad \text{at memory address 0x4ac}
    \end{align*}
    \noindent
    The array starts at address 0x4a0. Since hexadecimal is base-16, each digit represents 4 bits. Thus, array cells are 4 bytes apart.
\end{Def}

    
\newpage

\begin{Func}[Repeated-Squaring Algorithm]
    On input \( \alpha \in \mathbb{Z}_n \) and a non-negative integer \( e \), where \( e = (b_{\ell-1} \cdots b_0)_2 \) is the binary expansion of \( e \), the repeated-squaring algorithm computes \( \alpha^e \mod n \).

    \vspace{.5em}
    \noindent
    \textbf{Input:} \( \alpha \in \mathbb{Z}_n \), \( e \in \mathbb{Z}_{\geq 0} \) (binary)\\
    \textbf{Output:} \( \beta = \alpha^e \mod n \)\\
    
    \begin{algorithm}[H]
        \SetAlgoLined
        \SetKwProg{Fn}{Function}{:}{\KwRet{}}
        \Fn{\textit{RepeatedSquare}($\alpha, e$)}{
            $\beta \gets [1]_n$\;
            \For{$i \gets \ell - 1$ \textbf{down to} $0$}{
                $\beta \gets \beta^2 \mod n$\;
                \If{$b_i = 1$}{
                    $\beta \gets \beta \cdot \alpha \mod n$\;
                }
            }
            \KwRet{$\beta$}\;
        }
    \end{algorithm}

    \noindent\rule{\textwidth}{0.4pt}

    \noindent
    \textbf{Time Complexity:} \( O(\ell) \), where \( \ell \) is the number of bits in \( e \). This involves \( \ell \) squarings and at most \( \ell \) additional multiplications in \( \mathbb{Z}_n \).
    \textbf{Space Complexity:} \( O(\ell) \), where \( \ell \) is the number of bits in \( e \).
\end{Func}

\noindent
\textbf{Example 3.5.} Suppose $e = 37 = (100101)_2$. The above algorithm performs the following operations in this case:

\[
\begin{aligned}
    \beta &\gets [1] && // \text{0} \\
    \beta &\gets \beta^2, \ \beta \gets \beta \cdot \alpha && // \text{1} \\
    \beta &\gets \beta^2 && // \text{10} \\
    \beta &\gets \beta^2 && // \text{100} \\
    \beta &\gets \beta^2, \ \beta \gets \beta \cdot \alpha && // \text{1001} \\
    \beta &\gets \beta^2 && // \text{10010} \\
    \beta &\gets \beta^2, \ \beta \gets \beta \cdot \alpha && // \text{100101} 
\end{aligned}
\]

\vspace{1em}
\noindent 
We utilize this algorithm to compute \textbf{multiplicative inverses} and test, \textbf{quadratic residues} and \textbf{primality}. The following algorithms demonstrate 
such; Though in the following chapters, we will discuss more optimal solutions.

\newpage

\noindent

\begin{theo}[Computing Multiplicative Inverses in $\mathbb{Z}_p$]

    Let \( p \) be a prime and \( \alpha \in \mathbb{Z}_p^* \). The multiplicative inverse of \( \alpha \), denoted \( \alpha^{-1} \), can be computed using Euler's theorem. We have:

    \[
    \alpha^{p-1} = 1 \pmod{p}.
    \]

    \noindent
    Multiplying both sides by \( \alpha^{-1} \), we obtain:

    \[
    \alpha^{p-2} = \alpha^{-1} \pmod{p}.
    \]
    
    \noindent
    Thus, \( \alpha^{-1} \) is computed by raising \( \alpha \) to the power \( p-2 \).
\end{theo}

\begin{theo}[Testing Quadratic Residuosity]

    Let \( p \) be an odd prime and \( \alpha \in \mathbb{Z}_p^* \). To test whether \( \alpha \in (\mathbb{Z}_p^*)^2 \), we apply Euler's criterion:

    \[
    \alpha \in (\mathbb{Z}_p^*)^2 \iff \alpha^{(p-1)/2} = 1 \pmod{p}.
    \]
    
    \noindent
    Therefore, to determine whether \( \alpha \) is a quadratic residue, we raise \( \alpha \) to the power \( (p-1)/2 \).
\end{theo}



\begin{theo}[Computing Multiplicative Inverses in $\mathbb{Z}_p$]

    Let \( p \) be a prime and \( \alpha \in \mathbb{Z}_p^* \). The multiplicative inverse of \( \alpha \), denoted \( \alpha^{-1} \), can be computed using Euler's theorem. We have:

    \[
    \alpha^{p-1} = 1 \pmod{p}.
    \]
    
    \noindent
    Multiplying both sides by \( \alpha^{-1} \), we obtain:

    \[
    \alpha^{p-2} = \alpha^{-1} \pmod{p}.
    \]
    
    \noindent
    Thus, \( \alpha^{-1} \) is computed by raising \( \alpha \) to the power \( p-2 \).
\end{theo}

\noindent
Our next algorithm will optimize multiplication between two large integers. Our current method of multiplying $a,b$ and then dividing costs:
$Mul(a,b)=O(k\ell)$ where $k=||a||$ and $\ell=||b||$. Then for, $k=\ell=n$, we have $O(n^2)$, taking quadratic time. 

\newpage
\begin{theo}[Karatsuba’s Multiplication Algorithm]

    Let \( P \) and \( Q \) be large integers of length \( 2k \). Then \( P \cdot Q \) can be computed in \( O(n^{\log_2 3}) \):
    \[
    P \cdot Q = P_1 Q_1 \cdot 10^{2k} + \left[ (P_1 + P_2)(Q_1 + Q_2) - P_1 Q_1 - P_2 Q_2 \right] \cdot 10^k + P_2 Q_2.
    \]
    where subscripts \( 1 \) and \( 2 \) refer to the higher and lower orders, where $P_1 Q_1$ are $P_2 Q_2$ are computed once.
\end{theo}

\begin{Proof}[Karatsuba’s Multiplication Algorithm]

    \textbf{Step 1: Divide the numbers:} 
    Split the large integers \( P \) and \( Q \) into two halves:
    \[
    P = P_1 \cdot 10^k + P_2 \quad \text{and} \quad Q = Q_1 \cdot 10^k + Q_2,
    \]
    where \( P_1, P_2, Q_1, Q_2 \) represent the high and low parts of the numbers, and \( k \) is half the number of digits.\\

    \noindent
    \textbf{Step 2: Expand the product:} 
    The classical multiplication formula for \( P \cdot Q \) is:
    \[
    P \cdot Q = P_1 Q_1 \cdot 10^{2k} + (P_1 Q_2 + P_2 Q_1) \cdot 10^k + P_2 Q_2.
    \]
    We compute: \( P_1 Q_1 \) and \( P_2 Q_2 \), holding off on $(P_1 Q_2 + P_2 Q_1)$.\\

    \noindent
    \textbf{Step 3: Karatsuba’s Optimization:} 
    We reduce the number of multiplications by introducing the term: $ (P_1 + P_2)(Q_1 + Q_2)$.\\
    Expanding this, we have:
    \[
    (P_1 + P_2)(Q_1 + Q_2) = P_1 Q_1 + P_1 Q_2 + P_2 Q_1 + P_2 Q_2.
    \]
    Note, we already have \( P_1 Q_1 \) and \( P_2 Q_2 \). Let $P':= P_1 Q_1$ and $Q':= P_2 Q_2$. Then,
    \[
        (P_1 Q_2 + P_2 Q_1):= ((P_1 + P_2)(Q_1 + Q_2))-P'-Q'
    \]
    Where $((P_1 + P_2)(Q_1 + Q_2))$ is \textbf{one multiplication}, reducing multiplications to 3.\\
    

    \noindent
    \textbf{Step 4: Combine the results.} 
    Using the above observations, we can rewrite the product of \( P \cdot Q \) as:
    \[
    P \cdot Q = P_1 Q_1 \cdot 10^{2k} + \left[ (P_1 + P_2)(Q_1 + Q_2) - P_1 Q_1 - P_2 Q_2 \right] \cdot 10^k + P_2 Q_2.
    \]

    \noindent
    This results in only 3 recursive multiplications and achieves a time complexity of \( O(n^{\log_2 3}) \), which is approximately \( O(n^{1.585}) \), making it much faster than \( O(n^2) \).

\end{Proof}

\newpage 



