\section{Modular Arithmetic \& Residues}
\noindent
\textbf{Remember:} For $a\in\mathbb{R}$, \underline {$a\in[0,1)$ is a range}, i.e., including decimals from 0 to 1 (excluding 1).\\
\begin{Def}[Floor \& Ceiling]

    \label{def:floor_ceiling}

    For $x\in\mathbb{R}$ and $m,n\in\mathbb{Z}$. Functions map $\mathbb{R}\rightarrow\mathbb{Z}$,\\

    \noindent
    \textbf{Floor} $x$, $\floor{x}$, is the largest $m$ such that $m\leq x<m+\varepsilon$, where $\varepsilon\in[0,1)$.\\
    i.e., round down to the nearest integer.\\

    \vspace{-.5em}
    \noindent
    \textbf{Ceiling} $x$, $\ceil{x}$, is the smallest $n$ such that $n-\varepsilon<x\leq n$, where $\varepsilon\in[0,1)$.\\
    i.e., round up to the nearest integer.
\end{Def}
\noindent

\begin{Def}[Mod Operator]

    \label{def:mod_operator}

    Let $a,b\in\mathbb{Z}$, $b>0$: The remainder of $a$ divided by $b$. I.e., $a-b\left\lfloor\frac{a}{b}\right\rfloor$.\\

    \noindent
    \textbf{Denoted:} ``$a\bmod b$'' or ``$a\,\%\,b$''.
\end{Def}

\noindent
\textbf{Examples:} $8\bmod3=2$, and $5\bmod2=1$\\

\begin{Proof}[Mod Operator]
    \underline{The Division Algorithm (\ref{theo:division_algorithm}) only works for $b>0$.} To generalize for $b<0$,
    \begin{align*}
        a=bq+r \quad \quad    & \textit{ Given}                    \\
        a/b=q+r/b \quad \quad & \textit{ Divide both sides by $b$}
    \end{align*}
    \noindent
    We know $0\leq r<b$, dividing $b$ yielded $0\leq \dfrac{r}{b}<1$, so
    $$\dfrac{r}{b}\in[0,1)\in\mathbb{R}$$
    We notice $q=\left\lfloor\dfrac{a}{b}\right\rfloor$, as $q$ is the largest integer that fits into $a$, $b$ times.
\end{Proof}

\begin{Tip}
    $q=\left\lfloor\dfrac{a}{b}\right\rfloor$ is similar to integer division in programming, and $\dfrac{a}{b}=c$ implies $c\in\mathbb{R}$.
\end{Tip}

\newpage

\begin{theo}[Division Algorithm Extended]

    Let $a, b \in \mathbb{Z}$ with $b > 0$, and let $x \in \mathbb{R}$. Then there exist unique $q, r \in \mathbb{Z}$ such that $a = bq + r$ and $r \in [x, x + b)$.
\end{theo}

\noindent
$r \in [x, x + b)$ allows us to work with negative numbers and different intervals. Let's
try to build some intuition about division and remainders:

\begin{center}
    $a,b,r\in\mathbb{Z}$ and $S=\{r=a-bq:q\in\mathbb{Z}\}$, $a=6$, $b=2$:
\end{center}
\begin{center}
    (0)\qquad
    \begin{tabular}{c|cc}
        $x$ & $a-bx$                \\
        \hline
        0   & 0      & $=6-2\cdot0$ \\
        1   & 4      & $=6-2\cdot1$ \\
        2   & 2      & $=6-2\cdot2$ \\
        3   & 0      & $=6-2\cdot3$ \\
        \hline
        4   & -2     & $=6-2\cdot4$ \\
        5   & -4     & $=6-2\cdot5$ \\
        6   & -6     & $=6-2\cdot6$ \\
        7   & -8     & $=6-2\cdot7$ \\
    \end{tabular}
\end{center}

\noindent
Dividing two numbers varying the divisor:


\vfill
\noindent
(1)
\begin{minipage}{0.32\textwidth}
    \centering
    \begin{tabular}{c|c}
        \rowcolor{OliveGreen!10}$b$ & $3 \bmod b$ \\
        \hline
        1                           & 0           \\
        2                           & 1           \\
        3                           & 0           \\
        4                           & 3           \\
        5                           & 3           \\
        6                           & 3           \\
        7                           & 3           \\
        8                           & 3           \\
    \end{tabular}
\end{minipage}%
\begin{minipage}{0.32\textwidth}
    \centering
    \begin{tabular}{c|c}
        \rowcolor{OliveGreen!10}$b$ & $9 \bmod b$ \\
        \hline
        1                           & 0           \\
        2                           & 1           \\
        3                           & 0           \\
        4                           & 1           \\
        5                           & 4           \\
        6                           & 3           \\
        7                           & 2           \\
        8                           & 1           \\
        9                           & 0           \\
        10                          & 9           \\
    \end{tabular}
\end{minipage}%
\begin{minipage}{0.32\textwidth}
    \centering
    \begin{tabular}{c|c}
        \rowcolor{OliveGreen!10}$b$ & $7 \bmod b$ \\
        \hline
        1                           & 0           \\
        2                           & 1           \\
        3                           & 1           \\
        4                           & 3           \\
        5                           & 2           \\
        6                           & 1           \\
        7                           & 0           \\
        8                           & 7           \\
    \end{tabular}
\end{minipage}

\vfill
\newpage

\noindent
Grouping them by the remainder:\\
(2)
\begin{minipage}{0.32\textwidth}
    \centering
    \begin{tabular}{c|c}
        \rowcolor{OliveGreen!10}$r$ & $3 \bmod b$    \\
        \hline
        0                           & 1, 3           \\
        1                           & 2              \\
        3                           & 4, 5, 6, \dots \\
    \end{tabular}
\end{minipage}%
\begin{minipage}{0.32\textwidth}
    \centering
    \begin{tabular}{c|c}
        \rowcolor{OliveGreen!10}$r$ & $9 \bmod b$       \\
        \hline
        0                           & 1, 3, 9           \\
        1                           & 2, 4, 8           \\
        3                           & 5, 6, 7           \\
        9                           & 10, 11, 12, \dots \\
    \end{tabular}
\end{minipage}%
\begin{minipage}{0.32\textwidth}
    \centering
    \begin{tabular}{c|c}
        \rowcolor{OliveGreen!10}$r$ & $7 \bmod b$     \\
        \hline
        0                           & 1, 7            \\
        1                           & 2, 6            \\
        2                           & 5               \\
        3                           & 4               \\
        7                           & 8, 9, 10, \dots \\
    \end{tabular}
\end{minipage}


\noindent
Let's try the other way around.\\
(3)
\noindent
\begin{minipage}{0.32\textwidth}
    \centering
    \begin{tabular}{c|c}
        \rowcolor{OliveGreen!10}$a$ & $a \bmod 3$ \\
        \hline
        0                           & 0           \\
        1                           & 1           \\
        2                           & 2           \\
        3                           & 0           \\
        4                           & 1           \\
        5                           & 2           \\
        6                           & 0           \\
        7                           & 1           \\
        8                           & 2           \\
        9                           & 0           \\
    \end{tabular}
\end{minipage}%
\begin{minipage}{0.32\textwidth}
    \centering
    \begin{tabular}{c|c}
        \rowcolor{OliveGreen!10}$a$ & $a \bmod 9$ \\
        \hline
        0                           & 0           \\
        1                           & 1           \\
        2                           & 2           \\
        3                           & 3           \\
        \dots                       & \dots       \\
        9                           & 0           \\
        10                          & 1           \\
        11                          & 2           \\
        \dots                       & \dots       \\
        18                          & 0           \\
        19                          & 1           \\
    \end{tabular}
\end{minipage}%
\begin{minipage}{0.32\textwidth}
    \centering
    \begin{tabular}{c|c}
        \rowcolor{OliveGreen!10}$a$ & $a \bmod 7$ \\
        \hline
        0                           & 0           \\
        1                           & 1           \\
        2                           & 2           \\
        3                           & 3           \\
        4                           & 4           \\
        5                           & 5           \\
        6                           & 6           \\
        7                           & 0           \\
        8                           & 1           \\
        9                           & 2           \\
    \end{tabular}
\end{minipage}

\noindent
Grouping them by the remainder:\\
(4)
\begin{minipage}{0.32\textwidth}
    \centering
    \begin{tabular}{c|c}
        \rowcolor{OliveGreen!10}$r$ & $a \bmod 3$ \\
        \hline
        0                           & 0, 3, 6, 9  \\
        1                           & 1, 4, 7     \\
        2                           & 2, 5, 8     \\
    \end{tabular}
\end{minipage}%
\begin{minipage}{0.32\textwidth}
    \centering
    \begin{tabular}{c|c}
        \rowcolor{OliveGreen!10}$r$ & $a \bmod 9$ \\
        \hline
        0                           & 0, 9, 18    \\
        1                           & 1, 10, 19   \\
        2                           & 2, 11       \\
        3                           & 3, 12       \\
        4                           & 4, 13       \\
        5                           & 5, 14       \\
        6                           & 6, 15       \\
        7                           & 7, 16       \\
        8                           & 8, 17       \\
    \end{tabular}
\end{minipage}%
\begin{minipage}{0.32\textwidth}
    \centering
    \begin{tabular}{c|c}
        \rowcolor{OliveGreen!10}$r$ & $a \bmod 7$ \\
        \hline
        0                           & 0, 7        \\
        1                           & 1, 8        \\
        2                           & 2, 9        \\
        3                           & 3           \\
        4                           & 4           \\
        5                           & 5           \\
        6                           & 6           \\
    \end{tabular}
\end{minipage}


% Minipage for the first table with text on the left
\begin{minipage}{0.45\textwidth}

    (5) \textbf{Table with increments of 3}\\
    \begin{tabular}{c|ccc}
        $a$                       & $a+1$ & $a+2$ \\
        \hline
        \rowcolor{OliveGreen!10}0 & 1     & 2     \\
        \rowcolor{OliveGreen!10}3 & 4     & 5     \\
        \rowcolor{OliveGreen!10}6 & 7     & 8     \\
        9                         & 10    & 11    \\
        12                        & 13    & 14    \\
        15                        & 16    & 17    \\
        \dots                     & \dots & \dots \\
    \end{tabular}
\end{minipage}%
\hspace{0.05\textwidth} % Add some space between the text and the next section
\begin{minipage}{0.5\textwidth}
    What is multiplication but repeated addition?
    What is division but repeated subtraction?\\

    \noindent
    Column $a$ in (5)-(7) shows multiples of $b$, which is
    example (4) transposed (highlighted). We can think
    of the width of a table as $a$'s period.\\

    \noindent
    Add 10 to 8, yields numbers always ending in 8.
    Add 5 to 8, yields numbers ending in 3 or 8.\\
    Then there are periods like (3).\\

    \noindent
    We can see from the table (3), if we keep adding 3 to 2,
    we get 5, 8, 11, 14, etc.\\

\end{minipage}

\vspace{0.5cm} % Some vertical space between tables

% Second table without minipage
\noindent
(6) \textbf{Table with increments of 7}\\
\begin{tabular}{c|ccccccc}
    $a$                        & $a+1$                       & $a+2$                       & $a+3$ & $a+4$ & $a+5$ & $a+6$ \\
    \hline
    \rowcolor{OliveGreen!10}0  & 1                           & 2                           & 3     & 4     & 5     & 6     \\
    \cellcolor{OliveGreen!10}7 & \cellcolor{OliveGreen!10} 8 & \cellcolor{OliveGreen!10} 9 & 10    & 11    & 12    & 13    \\
    14                         & 15                          & 16                          & 17    & 18    & 19    & 20    \\
    21                         & 22                          & 23                          & 24    & 25    & 26    & 27    \\
    28                         & 29                          & 30                          & 31    & 32    & 33    & 34    \\
    35                         & 36                          & 37                          & 38    & 39    & 40    & 41    \\
    \dots                      & \dots                       & \dots                       & \dots & \dots & \dots & \dots \\
\end{tabular}

\vspace{0.5cm} % Some vertical space between tables

% Third table without minipage
\noindent
(7) \textbf{Table with increments of 9}\\
\begin{tabular}{c|ccccccccc}
    $a$                       & $a+1$ & $a+2$ & $a+3$ & $a+4$ & $a+5$ & $a+6$ & $a+7$ & $a+8$ \\
    \hline
    \rowcolor{OliveGreen!10}0 & 1     & 2     & 3     & 4     & 5     & 6     & 7     & 8     \\
    \rowcolor{OliveGreen!10}9 & 10    & 11    & 12    & 13    & 14    & 15    & 16    & 17    \\
    18                        & 19    & 20    & 21    & 22    & 23    & 24    & 25    & 26    \\
    27                        & 28    & 29    & 30    & 31    & 32    & 33    & 34    & 35    \\
    36                        & 37    & 38    & 39    & 40    & 41    & 42    & 43    & 44    \\
    45                        & 46    & 47    & 48    & 49    & 50    & 51    & 52    & 53    \\
    \dots                     & \dots & \dots & \dots & \dots & \dots & \dots & \dots & \dots \\
\end{tabular}

\noindent
We can represent these periods by $[x, x+b)$. Expanding the Division Algorithm (\ref{theo:division_algorithm}) beyond $b>0$, allows us to
represent intervals no matter where we start on the number line.\\

\noindent
We formally group (5)-(7)'s column headers into classes, which we call \underline{\textbf{\textit{residues}.}}

\newpage

\begin{Def}[Residue]

    \label{def:residue}

    Let $a,n\in\mathbb{Z}$, $n>0$.\\

    \noindent
    Set $R=\{a\bmod n: n\in\mathbb{Z}, n\neq0\}$ produces remainders $r\in[0,n-1]$.\\
    Each remainder $r$ is a \underline{\textbf{residue}} of $a$ modulo $n$.
\end{Def}

\begin{Def}[Residue Class]

    \label{def:residue_class}

    The set of numbers produced by a residue.\\

    \noindent
    \textbf{Denoted:} $[a]_n$ or $a (\bmod\, n)$, $a$ is the residue under modulo $n$.
\end{Def}
\begin{Note}
    \textbf{Note:} If modulo $n$ has already been defined, $[a]_n$, then $[a]$ might be used.
\end{Note}
\begin{Def}[Representative]

    \label{def:representative}

    If $x\in [a]$, $x$ is a representative of $[a]$.
\end{Def}





