\section{Ring Theory}
We will primarily focus on \textbf{ideals} and the behavior of primes; Though
to understand ideals, is to understand \textbf{groups }, \textbf{rings}, and \textbf{fields}.

\begin{Def}[Group]

    A \textit{group} is a set $G$ that is closed under one operation, say `$*$', that satisfies four properties:
    \begin{itemize}
        \item \textbf{Closure:} For all $a,b\in G$, $a*b\in G$.
        \item \textbf{Associativity:} For all $a,b,c\in G$, $(a*b)*c=a*(b*c)$.
        \item \textbf{Identity:} There exists an element $e\in G$ such that for all $a\in G$, $a*e=e*a=a$.
        \item \textbf{Inverse:} $\forall a\,\exists\, a^{-1}\in G$, such that $a*a^{-1}=a^{-1}*a$ equates to the identity.
    \end{itemize}
\end{Def}
\newpage

\noindent
\textbf{Examples:} The following are groups:
\begin{itemize}
    \item Set $S=\{-1,1\}$ closed under multiplication.
          \begin{itemize}
              \item \textbf{Closure:} $-1\cdot-1=1\in S$.
              \item \textbf{Associativity:} $(-1\cdot1)\cdot-1=1\cdot-1=-1$ and $-1\cdot(1\cdot-1)=-1\cdot1=-1$.
              \item \textbf{Identity:} 1, as $1 \cdot -1=-1\cdot1=-1$.
              \item \textbf{Inverse:} -1 as $-1\cdot-1=1=$ the identity.
          \end{itemize}

    \item Set $I=\mathbb{Z}$ closed under addition.
          \begin{itemize}
              \item \textbf{Closure:} $a+b\in I$ for all $a,b\in I$.
              \item \textbf{Associativity:} $(a+b)+c=a+(b+c)$ for all $a,b,c\in I$.
              \item \textbf{Identity:} 0, as $a+0=0+a=a$ for all $a\in I$.
              \item \textbf{Inverse:} $-a$ for all $a\in I$, as $a+(-a)=(-a)+a=0$.
          \end{itemize}
\end{itemize}

\begin{Def}[Abelian Group]

    An \textit{Abelian group} is a group that also satisfies the commutative property, i.e., for all $a,b\in G$, $a*b=b*a$.
    for some operation `$*$'.

\end{Def}

\begin{Def}[Ring]

    A \textit{ring} is a non-empty set $R$ that is closed under additive (+) and multiplicative $(\cdot)$ operations, such that:
    \begin{itemize}
        \item \textbf{Additive Group:} $(R)$ is an Abelian group.
        \item \textbf{Multiplicative Closure:} For all $a,b\in R$, $a\cdot b\in R$.
        \item \textbf{Distributive Property:} For all $a,b,c\in R$, $a\cdot(b+c)=a\cdot b+a\cdot c$ and $(a+b)\cdot c=a\cdot c+b\cdot c$.
    \end{itemize}
\end{Def}

\noindent
\textbf{Examples:} $\mathbb{Z}$, $\mathbb{Q}$, $\mathbb{R}$, and $\mathbb{C}$ are all rings standard addition and multiplication.\\

\begin{Note}
    \textbf{Note:} Operations aren't literally addition and multiplication. For example,
    the set of $2\times2$ matrices with $\mathbb{R}$ entries forms a ring.
\end{Note}
\begin{Tip}
    Numbers and symbols are just placeholders for the concepts they represent.
    1,2 or ($\div$) don't have inherent properties; they are just symbols, changing meaning
    in different contexts.
\end{Tip}

\newpage

\begin{Def}[Ideal]

    An \textit{ideal} $I$, is a special subset of a ring $R$, such that for all $a,b\in I$ and $r\in R$:
    \begin{itemize}
        \item \textbf{Additive:} $a+b\in I$.
        \item \textbf{Multiplicative under the ring:} $a\cdot r\in I$ or $r\cdot a\in I$.
        \item \textbf{Additive inverse:} $-a\in I$.
        \item \textbf{Additive identity:} $a,a'\in I$ such that $a+a'=a'+a=a'$.
    \end{itemize}
\end{Def}

\noindent
\textbf{Example:} The set of all multiples of 2, $2\mathbb{Z}$, is an ideal of $\mathbb{Z}$.
\begin {itemize}
\item \textbf{Additive:} $(2\cdot a)+(2\cdot b)=2(a+b)\in 2\mathbb{Z}$.
\item \textbf{Multiplicative:} $(2\cdot a)\cdot r=2(a\cdot r)\in 2\mathbb{Z}$.
\item \textbf{Additive inverse:} $-2\in 2\mathbb{Z}$.
\item \textbf{Additive identity:} $0\in 2\mathbb{Z}$.
\end{itemize}

% why is Z not considered an ideal?

\begin{Def}[Field]

    A \textit{field} is a ring $\mathbb{F}$ with additional properties:
    \begin{itemize}
        \item \textbf{Additive Structure:} $(\mathbb{F},+)$ forms an Abelian group.
        \item \textbf{Multiplicative Structure:} $(\mathbb{F},\cdot)$ forms an Abelian group \underline{excluding 0}:
        \item \textbf{Distributive:} For all $a,b,c\in\mathbb{F}$, $a\cdot(b+c)=a\cdot b+a\cdot c$.
    \end{itemize}
\end{Def}

\noindent
\textbf{Example:} $\mathbb{Q}$, the set of rational numbers:
\begin{itemize}
    \item \textbf{Multiplicative identity:} $1\in\mathbb{Q}$ as $1\cdot a=a\cdot 1=a$ for all $a\in\mathbb{Q}$.
    \item \textbf{Multiplicative inverse:} $a^{-1}=\frac{1}{a}$ as $a\cdot\frac{1}{a}=1$.
    \item \textbf{Excludes 0:} As $0$ has no multiplicative inverse, i.e., $\frac{1}{0}$ is undefined.
\end{itemize}
\begin{Tip}
    A \textbf{group} defines operations, an \textbf{abelian group} ensures
    commutativity, a \textbf{ring} has an abelian group (+), multiplication ($\cdot$), and distribution,
    an \textbf{ideal} $I\subseteq  R$ ring, such that\\ $a\in I,r\in R, a\cdot r\in I$, and a \textbf{field} is a ring
    excluding 0 in its multiplicative abelian group.
\end{Tip}

\newpage

\section{Ideals \& Primality}
We will use $\mathbb{Z}$ as an ideal to explore the behavior of primes and divisibility.

\begin{Def}[Generator]

    An element or set of elements that can be used to \textit{generate} a structure by repeated application of that structure's operations.
\end{Def}


\begin{Def}[Integer Ideal Generator]

    The ideal generated by an integer \( a \) in \( \mathbb{Z} \), denoted $a\mathbb{Z}$, is the set of all multiples of \( a \):
    \begin{equation*}
        a\mathbb{Z} = \{ a \cdot x : x \in \mathbb{Z} \}= \{ \ldots, -2a, -a, 0, a, 2a, \ldots \}.
    \end{equation*}

    \noindent
    Also Denoted: \underline{$\langle a \rangle$ when the generator is clear.}
\end{Def}

\noindent
\textbf{Example:} The ideal generated by 2, $\langle 2 \rangle=\{\ldots,-4,-2,0,2,4,\ldots\}$.

\begin{Proof}[Proof that $a\mathbb{Z}$ is an Ideal]

    Let $a\mathbb{Z}$ be the ideal generated by $a \in \mathbb{Z}$, and let $az, az' \in a\mathbb{Z}$,$z'' \in \mathbb{Z}$, and $r \in \mathbb{R}$.
    \begin{itemize}
        \item \textbf{Additive Closure:} $az + az' = a(z + z') \in a\mathbb{Z}$.
        \item \textbf{Multiplicative Closure:} $az \cdot r = a(z \cdot r)$, then $(z\cdot r)\in\mathbb{Z}$ therefore $a(z \cdot r)\in a\mathbb{Z}$.
        \item \textbf{Additive Inverses:} $-az = a(-z) \in a\mathbb{Z}$.
        \item \textbf{Additive Identity:} $a \cdot 0 = 0 \in a\mathbb{Z}$.

    \end{itemize}
    Therefore, $a\mathbb{Z}$ is an ideal of $\mathbb{Z}$.
\end{Proof}


\begin{Def}[Principal Ideal]

    For ring $R$ and $a\in R$, if $\langle a \rangle = \{ r \cdot a : r \in R \}$
    and $\langle a \rangle$ is an ideal of $R$, then $\langle a \rangle$ is a \textit{principal ideal}.
\end{Def}

\noindent
Since $\mathbb{Z}$ forms a ring, for $a\in\mathbb{Z}$, $\langle a \rangle$ is a principal ideal of $\mathbb{Z}$. \underline{It also follows that $\langle a \rangle \subseteq \mathbb{Z}$.}

\newpage

\begin{Def}[Ideal Operations]

    \label{def:ideal_operations}

    Let $I$ and $J$ be ideals of a ring $R$.
    \begin{itemize}
        \item \textbf{Sum:} The sum of two ideals $I + J$ is defined as:
              \[
                  I + J = \{ i + j : i \in I, j \in J \}.
              \]
              Since $I$ and $J$ are both have multiplicative closures of $R$, their sum is too.\\
              \[
                  (i\cdot r) \in I \text{ and } (j\cdot r) \in J \text{ then } (i\cdot r) + (j\cdot r) = (i+j)\cdot r \in I+J.
              \]
        \item \textbf{Product:} The product of two ideals $I \cdot J$ is defined as:
              \[
                  I \cdot J = \left\{ \sum i \cdot j : i \in I, j \in J \right\}.
              \]
              We need $\sum$ to show additive closure. We represent
              our product as sums alike $I+J$:\\
              For $i'\in I$:
              \[
                  (i\cdot j) + (i'\cdot j) = (i+i')\cdot j = i\cdot j \in I\cdot J.
              \]

    \end{itemize}
\end{Def}

\noindent
This follows from the properties of ideals in $\mathbb{Z}$ and can be generalized to any ring $R$.\\

\noindent
\textbf{Example:} Consider ideals in $\mathbb{Z}$:

\[
    I = 2\mathbb{Z} = \{ \ldots, -4, -2, 0, 2, 4, \ldots \} \quad \text{(the even integers)}
\]
and
\[
    J = 3\mathbb{Z} = \{ \ldots, -6, -3, 0, 3, 6, \ldots \} \quad \text{(the multiples of 3)}.
\]

\noindent
The product $I \cdot J$ is not just the set of all individual products like $2 \cdot 3 = 6$. Instead, it is the set of all sums of products of elements from $I$ and $J$, including sums like:

\[
    2 \cdot 3 + (-2) \cdot 3 = 6 - 6 = 0
\]
or
\[
    2 \cdot 3 + 4 \cdot 3 = 6 + 12 = 18.
\]

Thus, the product of $I$ and $J$ is:
\[
    I \cdot J = \{ \ldots, -18, -12, -6, 0, 6, 12, 18, \ldots \} = 6\mathbb{Z}.
\]
Therefore, the product of $2\mathbb{Z}$ and $3\mathbb{Z}$ is $6\mathbb{Z}$, the set of multiples of 6.
Illustrating $I \cdot J$ as the sums of products ensures the additive and multiplicative closure properties of ideals.
\newpage

\begin{theo}[Ideal Properties]

    \label{theo:ideal_properties}

    For ideals in the integers $\mathbb{Z}$, and all $a, b \in \mathbb{Z}$:
    \begin{itemize}
        \item $b \in a\mathbb{Z}$ if and only if $a \mid b$.
        \item For every ideal $I \subseteq \mathbb{Z}$, $b \in I$ if and only if $b\mathbb{Z} \subseteq I$.
        \item Combining the above observations: $b\mathbb{Z} \subseteq a\mathbb{Z}$ if and only if $a \mid b$.
    \end{itemize}
\end{theo}

\begin{Proof}[Proof of Ideal Properties]

    \begin{itemize}
        \item $b \in a\mathbb{Z}$, let $a$ be the smallest positive integer, then $b$ must be 0, $a$, or some
        multiple of $a$, thus $a \mid b$. If $a\mid b$, then $b\in a\mathbb{Z}$, as $a\mathbb{Z}$ generates multiples of $a$.
        \item $b \in I$, then $b\mathbb{Z} \subseteq I$ as $I$ upholds multiplicative closure. I.e., $q\in I$ then $bq\in I$.
        \item $a\mid b$, then $b\in a\mathbb{Z}$, $a\mathbb{Z}$ is an ideal, thus $b\mathbb{Z} \subseteq a\mathbb{Z}$. If $b\mathbb{Z} \subseteq a\mathbb{Z}$, then $b\in a\mathbb{Z}$, and $a\mid b$.
    \end{itemize}
\end{Proof}


\begin{theo}[Ideal Generator Existance of $\mathbb{Z}$]

    \label{theo:ideal_generator}

    Let \( I \) be an ideal of \( \mathbb{Z} \). Then there exists a unique non-negative integer \( d \) such that \( I = d\mathbb{Z} \).
\end{theo}

\begin{Proof}[Proof of Generator Ideal equality of \( \mathbb{Z} \)]
    \begin{itemize}
        \item \textbf{Existence:} $I=\{0\}$, then $d = 0$.
        \item $\mathbf{I \neq \{0\}:}$ Let $d$ be the smallest positive integer in $I$. 
        If $a\in I$, then $d\mid a$, because  $a = dq + r$ for some $q,r\in\mathbb{Z}$, where $0\leq r < d$ (\ref{theo:division_algorithm}).
        Since $d$ is the smallest positive integer, $r = 0$, hence $d\mid a$.
        \item $\mathbf{I\subseteq d\mathbb{Z}}$, as $d\mid a$ and $a\in I$ (\ref{theo:ideal_properties}).
        \item \textbf{Uniqueness:} Let $d'$ be another non-negative integer. If $d'\mathbb{Z}=d\mathbb{Z}$,
        then $d\mid d'$ and $d'\mid d$. Thus, $d=\pm d'$ (\ref{theo:properties_of_divisibility}). Since, $d'\geq0$, $d=d'$.
    \end{itemize}
    

\end{Proof}



