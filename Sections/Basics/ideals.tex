\section{Ring Theory}
We will primarily focus on \textbf{ideals} and the behavior of primes; Though
to understand ideals, is to understand \textbf{groups }, \textbf{rings}, and \textbf{fields}.

\begin{Def}[Group]

    A \textit{group} is a set $G$ that is closed under one operation, say `$*$', that satisfies four properties:
    \begin{itemize}
        \item \textbf{Closure:} For all $a,b\in G$, $a*b\in G$.
        \item \textbf{Associativity:} For all $a,b,c\in G$, $(a*b)*c=a*(b*c)$.
        \item \textbf{Identity:} There exists an element $e\in G$ such that for all $a\in G$, $a*e=e*a=a$.
        \item \textbf{Inverse:} $\forall a\,\exists\, a^{-1}\in G$, such that $a*a^{-1}=a^{-1}*a$ equates to the identity.
    \end{itemize}
\end{Def}
\newpage

\noindent
\textbf{Examples:} The following are groups:
\begin{itemize}
    \item Set $S=\{-1,1\}$ closed under multiplication.
          \begin{itemize}
              \item \textbf{Closure:} $-1\cdot-1=1\in S$.
              \item \textbf{Associativity:} $(-1\cdot1)\cdot-1=1\cdot-1=-1$ and $-1\cdot(1\cdot-1)=-1\cdot1=-1$.
              \item \textbf{Identity:} 1, as $1 \cdot -1=-1\cdot1=-1$.
              \item \textbf{Inverse:} -1 as $-1\cdot-1=1=$ the identity.
          \end{itemize}

    \item Set $I=\mathbb{Z}$ closed under addition.
          \begin{itemize}
              \item \textbf{Closure:} $a+b\in I$ for all $a,b\in I$.
              \item \textbf{Associativity:} $(a+b)+c=a+(b+c)$ for all $a,b,c\in I$.
              \item \textbf{Identity:} 0, as $a+0=0+a=a$ for all $a\in I$.
              \item \textbf{Inverse:} $-a$ for all $a\in I$, as $a+(-a)=(-a)+a=0$.
          \end{itemize}
\end{itemize}

\begin{Def}[Abelian Group]

    An \textit{Abelian group} is a group that also satisfies the commutative property, i.e., for all $a,b\in G$, $a*b=b*a$.
    for some operation `$*$'.

\end{Def}

\begin{Def}[Ring]

    A \textit{ring} is a non-empty set $R$ that is closed under additive (+) and multiplicative $(\cdot)$ operations, such that:
    \begin{itemize}
        \item \textbf{Additive Group:} $(R)$ is an Abelian group.
        \item \textbf{Multiplicative Closure:} For all $a,b\in R$, $a\cdot b\in R$.
        \item \textbf{Distributive Property:} For all $a,b,c\in R$, $a\cdot(b+c)=a\cdot b+a\cdot c$ and $(a+b)\cdot c=a\cdot c+b\cdot c$.
    \end{itemize}
\end{Def}

\noindent
\textbf{Examples:} $\mathbb{Z}$, $\mathbb{Q}$, $\mathbb{R}$, and $\mathbb{C}$ are all rings standard addition and multiplication.\\

\begin{Note}
    \textbf{Note:} Operations aren't literally addition and multiplication. For example,
    the set of $2\times2$ matrices with $\mathbb{R}$ entries forms a ring.
\end{Note}
\begin{Tip}
    Numbers and symbols are just placeholders for the concepts they represent.
    1,2 or ($\div$) don't have inherent properties; they are just symbols, changing meaning
    in different contexts.
\end{Tip}

\newpage

\begin{Def}[Ideal]

    An \textit{ideal} $I$, is a special subset of a ring $R$, such that for all $a,b\in I$ and $r\in R$:
    \begin{itemize}
        \item \textbf{Additive:} $a+b\in I$.
        \item \textbf{Multiplicative under the ring:} $a\cdot r\in I$ or $r\cdot a\in I$.
        \item \textbf{Additive inverse:} $-a\in I$.
        \item \textbf{Additive identity:} $a,a'\in I$ such that $a+a'=a'+a=a'$.
    \end{itemize}
\end{Def}

\noindent
\textbf{Example:} The set of all multiples of 2, $2\mathbb{Z}$, is an ideal of $\mathbb{Z}$.
\begin {itemize}
\item \textbf{Additive:} $(2\cdot a)+(2\cdot b)=2(a+b)\in 2\mathbb{Z}$.
\item \textbf{Multiplicative:} $(2\cdot a)\cdot r=2(a\cdot r)\in 2\mathbb{Z}$.
\item \textbf{Additive inverse:} $-2\in 2\mathbb{Z}$.
\item \textbf{Additive identity:} $0\in 2\mathbb{Z}$.
\end{itemize}

% why is Z not considered an ideal?

\begin{Def}[Field]

    A \textit{field} is a ring $\mathbb{F}$ with additional properties:
    \begin{itemize}
        \item \textbf{Additive Structure:} $(\mathbb{F},+)$ forms an Abelian group.
        \item \textbf{Multiplicative Structure:} $(\mathbb{F},\cdot)$ forms an Abelian group \underline{excluding 0}:
        \item \textbf{Distributive:} For all $a,b,c\in\mathbb{F}$, $a\cdot(b+c)=a\cdot b+a\cdot c$.
    \end{itemize}
\end{Def}

\noindent
\textbf{Example:} $\mathbb{Q}$, the set of rational numbers:
\begin{itemize}
    \item \textbf{Multiplicative identity:} $1\in\mathbb{Q}$ as $1\cdot a=a\cdot 1=a$ for all $a\in\mathbb{Q}$.
    \item \textbf{Multiplicative inverse:} $a^{-1}=\frac{1}{a}$ as $a\cdot\frac{1}{a}=1$.
    \item \textbf{Excludes 0:} As $0$ has no multiplicative inverse, i.e., $\frac{1}{0}$ is undefined.
\end{itemize}
\begin{Tip}
    A \bt{group} defines operations, an \bt{abelian group} ensures
    commutativity, a \bt{ring} has an abelian group (+), multiplication ($\cdot$), and distribution,
    an \bt{ideal} $I\subseteq  R$ ring, such that\\ $a\in I,r\in R, a\cdot r\in I$, and a \bt{field} is a ring
    excluding 0 in its multiplicative abelian group.
\end{Tip}

\newpage

\section{Ideals \& Primality}
We will use $\mathbb{Z}$ as an ideal to explore the behavior of primes and divisibility.

\begin{Def}[Generator]

    An element or set of elements that can be used to \textit{generate} a structure by repeated application of that structure's operations.
\end{Def}


\begin{Def}[Integer Ideal Generator]

    The ideal generated by an integer \( a \) in \( \mathbb{Z} \), denoted $a\mathbb{Z}$, is the set of all multiples of \( a \):
    \begin{equation*}
        a\mathbb{Z} = \{ a \cdot x : x \in \mathbb{Z} \}= \{ \ldots, -2a, -a, 0, a, 2a, \ldots \}.
    \end{equation*}

    \noindent
    Also Denoted: \underline{$\langle a \rangle$ when the generator is clear.}
\end{Def}

\noindent
\textbf{Example:} The ideal generated by 2, $\langle 2 \rangle=\{\ldots,-4,-2,0,2,4,\ldots\}$.

\begin{Proof}[Proof that $a\mathbb{Z}$ is an Ideal]

    Let $a\mathbb{Z}$ be the ideal generated by $a \in \mathbb{Z}$, and let $az, az' \in a\mathbb{Z}$,$z'' \in \mathbb{Z}$, and $r \in \mathbb{R}$.\\
    \begin{itemize}
        \item \textbf{Additive Closure:} $az + az' = a(z + z') \in a\mathbb{Z}$.
        \item \textbf{Multiplicative Closure:} $az \cdot r = a(z \cdot r)$, then $(z\cdot r)\in\mathbb{Z}$ therefore $a(z \cdot r)\in a\mathbb{Z}$.
        \item \textbf{Additive Inverses:} $-az = a(-z) \in a\mathbb{Z}$.
        \item \textbf{Additive Identity:} $a \cdot 0 = 0 \in a\mathbb{Z}$.

    \end{itemize}
    Therefore, $a\mathbb{Z}$ is an ideal of $\mathbb{Z}$.
\end{Proof}


\begin{Def}[Principal Ideal]

    For ring $R$ and $a\in R$, if $\langle a \rangle = \{ r \cdot a : r \in R \}$
    and $\langle a \rangle$ is an ideal of $R$, then $\langle a \rangle$ is a \textit{principal ideal}.
\end{Def}

Namely we can use this to generate principal ideals in $\mathbb{Z}$.