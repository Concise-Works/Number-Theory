\chapter{Basic properties of Integers}

\section{Divisibility and primality}

\bt{``$a$ divides $b$''}, i.e., $\left(\frac{b}{a}\right)$, means \bt{$b$} is reached by \bt{$a$}, when \underline{\bt{$a$} is multiplied by \textbf{some integer}.}

\begin{Def}[Division]

    Let $a,b,x\in\mathbb{Z}$: $\left(\frac{b}{a}\right)$ means ``$b=ax$''.\\

    \noindent
    \underline{\textbf{Denoted:} $a|b$,}\\
    $\qquad$ read ``$a$ divides $b$,'' and ``$a$ doesn't divide $b$'' is, $a\nmid b$.\\
    \noindent
\end{Def}

\noindent
\textbf{Examples:}
\begin{itemize}
    \item $3\mid 6$ because $6=3\cdot2$.
    \item $3\nmid5$ because $5\neq3\cdot x$ for any $x\in\mathbb{Z}$.
    \item $2\mid 0$ because $0=2\cdot0$.
    \item $0\nmid2$ because $2\neq0\cdot x$ for any $x\in\mathbb{Z}$.
\end{itemize}


\begin{Note}
    \textbf{Note:} $a,b,x\in\mathbb{Z}$ for, ``$\left(\frac{b}{a}\right)$'' or ``$b = ax$'' are labeled, $a$: \bt{divisor}, $b$: \bt{dividend}, $x$: \bt{quotient}.
\end{Note}

\begin{Tip}
    Many problems will involve manipulating this ``$b=ax$'' equation. Whether
    it's substituting $b$ for $ax$ or vice-versa, or adding/subtracting/multiplying/dividing ``$b=ax$''
    to itself to reveal some property.\\

    \noindent
    Many definitions and theorems will relate to each other or build off one another.
    It's crucial to understand what concepts mean rather than memorizing them. This means
    the ability to derive theorems or definitions from scratch, based on intuitive
    understanding of the content.
\end{Tip}

\newpage

\noindent
Observe the following:
\begin{theo}[Properties of Divisibility]

    \textbf{Theorem 1.1.} For all $a, b, c \in \mathbb{Z}$:

    \begin{itemize}
        \item[(i)] ``$a \mid a$'', ``$1 \mid a$'', and ``$a \mid 0$''
        \item[(ii)] ``$0 \mid a$'' $\Longleftrightarrow$ ``$a = 0$''
        \item[(iii)] ``$a \mid b$'' $\Longleftrightarrow$ ``$-a \mid b$'' $\Longleftrightarrow$ ``$a \mid -b$''
        \item[(iv)] ``$a \mid b$'' $\land$ ``$a \mid c$'' $\Longrightarrow$ ``$a \mid (b + c)$''
        \item[(v)] ``$a \mid b$'' $\land$ ``$b \mid c$'' $\Longrightarrow$ ``$a \mid c$''
    \end{itemize}
\end{theo}

\noindent
Try to prove these properties before reading the proof below.
\noindent
\begin{Proof}[Theorem 1.1]
    \textit{\textbf{Proof.}} For all $a, b, x,y \in \mathbb{Z}$:
    \begin{itemize}
        \item[(i)]  \begin {itemize}
        \item ``$a\mid a$'' means ``$a = ax$'', choosing $x = 1$ always satisfies.
        \item ``$1\mid a$'' because $a = 1\cdot a$
        \item ``$a\mid 0$'' because $0 = a\cdot 0$
    \end{itemize}
    \item[(ii)] \begin {itemize}
    \item If ``$0\mid a$'' then ``$a = 0\cdot x$'', 0 times any integer is 0, so $a = 0$
    \item If ``$a = 0$'' then ``$0 = 0\cdot x$'', $x$ can be any integer.
    \end {itemize}

    \noindent
    \item[(iii)]
    Proving $a\mid b \Longleftrightarrow -a\mid b$:
    \begin{itemize}
        \item If ``$a\mid b$'' then ``$b = ax = (-a)(-x)$'', $-x$ is some integer, say $x'$.\\
              So ``$b = (-a)x'$'' then ``$-a\mid b$''
        \item If ``$-a\mid b$'' then ``$b = (-a)x$'', choose $x$ to be some negative integer.\\
    \end{itemize}
    \vspace{-1em}
    Proving $-a\mid b \Longleftrightarrow a\mid -b$:
    \begin{itemize}
        \item If ``$-a\mid b$'' then ``$b = (-a)x$'', choose $x$ positive integer.
        \item If ``$a\mid -b$'' then ``$-b = ax$'', choose $x$ to be some negative integer.
    \end{itemize}
    \item[(iv)] If ``$a\mid b$'' and ``$a\mid c$'' then ``$b = ax$'' and ``$c = ay$'' add both equations, ``$b+c = ax+ay$''
    factor, ``$b+c = a(x+y)$'', $(x+y)$ is some integer, so ``$a\mid (b+c)$''

    \item[(v)] If ``$a\mid b$'' and ``$b\mid c$'' then ``$b = ax$'' and ``$c = by$'' substitute $b$ in $c$, ``$c = (ax)y$''\\
    shift terms, ``$c = a(xy)$'', $(xy)$ is some integer, so ``$a\mid c$''.
    \end{itemize}

\end{Proof}

\newpage

\begin {theo}[Reflexive Divisibility]

For all $a, b \in \mathbb{Z}$: ``$a \mid b$'' $\land$ ``$b \mid a$'' $\Longleftrightarrow$ ``$a = \pm b$''. Additionally, ``$a \mid 1$'' $\Longleftrightarrow$ ``$a = \pm 1$''.
\end{theo}

\begin{Proof}[Theorem 1.2]
    \textit{\textbf{Proof.}} For all $a, b, x,y \in \mathbb{Z}$:\\

    \noindent
    Proving ``$a\mid b$'' $\land$ ``$b\mid a$'' $\Longrightarrow$ ``$a = \pm b$'':
    \begin{align*}
        a\mid b \quad a\mid b \quad \quad & \textit{ Given}                      \\
        b = ax \quad a = by         \quad & \textit{ Definition of Division}     \\
        ab = (ax) (by)              \quad & \textit{ Multiplying both equations} \\
        ab = (ab)(xy)               \quad & \textit{ Shift terms}                \\
        1 = xy                      \quad & \textit{ Divide both sides by } ab
    \end{align*}

    \noindent
    $x$ and $y$ are integers, so ``$x = y = 1$''. Substitute $x$ and $y$,
    \hspace{-3em}
    \begin{align*}
        b = a(1) \quad a = b(1) \quad & \textit{ Substitute}              \\
        a = b \quad                   & \textit{ Simplify } \hspace*{9em}
    \end{align*}
    $x$ or $y$ could be $\pm$, so ``$a = \pm b$''. Now ``$a=\pm b$'' $\Longrightarrow$ ``$a\mid b$'' and ``$b\mid a$''.
    From Theorem 1.1, we can use (i) to show ``$a\mid a$'' substitute $b$ in for $a$, ``$a\mid b$'' or $b\mid a$.\\

    \noindent
    Proving ``$a\mid 1$'' $\Longrightarrow$ ``$a = \pm 1$'':
    \begin{align*}
        a\mid 1 \quad \quad  & \textit{ Given}                  \\
        1 = ax \quad \quad   & \textit{ Definition of Division} \\
        1 = a(1) \quad \quad & \textit{ Simplify}               \\
    \end{align*}
    $a$ must be 1, $x$ could be $\pm$, so ``$a = \pm 1$'' then ``$a\mid \pm 1$'' so ``$a\mid 1$''.
\end{Proof}

\begin{Def}[Cancellation Law]

    Let $a,b,c\in\mathbb{Z}$: If ``$ab=ac$'' and ``$a\neq 0$'' then ``$b=c$''.

\end{Def}

\noindent
I.e., given ``$b=c$'' multiplying both sides by $a$ yields ``$ab=ac$'', still $b=c$.

\newpage

\begin{Def}[Prime Numbers]

    $p\in\mathbb{Z}$ is prime if $p\neq 0$ and $p$ has no divisors other than $1$ and $p$.
\end{Def}
\underline{We will \textbf{only} consider positive prime numbers,} in this text. Examples of primes are:
$$2,3,5,7,11,13,17,\dots$$

\begin{Def}[Composite Numbers]

    $n,a,b\in\mathbb{Z}$ is composite if $n=ab$ and $1<a<n$ and $1<b<n$.
\end{Def}
I.e., a composite number is a number can be factor into two integers, other than $1$ and itself.

\noindent
\textbf{Examples:}
\begin{itemize}
    \item $4$ is composite because $4=2\cdot2$.
    \item $6$ is composite because $6=2\cdot3$.
    \item $10$ is composite because $10=2\cdot5$.
\end{itemize}

\noindent
Briefly observe the following:
\begin{theo}[Division Algorithm]

    For all $a,b\in\mathbb{Z}$, $b\neq0$, there exists unique $q,r\in\mathbb{Z}$ such that $a=bq+r$ and $0\leq r<|b|$.
\end{theo}

\noindent
To dissect, for all $a,b,q,r\in\mathbb{Z}$, $b\neq0$, $q$ and $r$ exist uniquely such that:

\begin{center}
    \HUGE{``\textcolor{red}{$a$}=\textcolor{OliveGreen}{$b$}\textcolor{orange}{$q$}+\textcolor{blue}{$r$}''}\\
    \vspace*{.5em}
    \LARGE{\textcolor{red}{\textbf{Dividend}} = \textcolor{OliveGreen}{\textbf{Divisor}} $\cdot$ \textcolor{orange}{\textbf{Quotient}} + \textcolor{blue}{\textbf{Remainder}}}\\
    \Large{\textcolor{OliveGreen}{$b$} fits into \textcolor{red}{$a$} \textcolor{orange}{$q$} times with \textcolor{blue}{$r$} left over.}
\end{center}



